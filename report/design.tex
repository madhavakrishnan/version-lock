\section{Design}
\label{s:des}

In this section we explain our design goals followed by the \sys design.

\subsection{Design Goals}
\label{s:des:goals}

\PP{Avoid read-modify writes for read locks}
In the PCC techniques, acquiring lock involves modifying the shared memory
metadata, this is a problem especially for readers. Essentially, the read
operations becomes read-modify-write operation which incurs high overhead. 
In \sys, we aim to make the read lock acquire and release operation without any
writes to the shared memory.

\PP{Avoid cacheline contention for read locks}
In the lock-free programming, contention on a single cacheline causes cacheline
bouncing~\cite{} which is one of the chief performance and scalability
inhibitor for both read and write operations. In \sys, we aim to minimize the
effects of cacheline bouncing by making writes exclusive. 

\PP{Achieve high scalability}
Locks need to be scalable across many CPU cores in order foster wider adoption.
We aim to make \sys highly scalable for both read and write operations.

\PP{Simple lock APIs and interface}
We aim to expose \sys with some simple APIs to achieve better programmability
and possibly to avoid significant changes to the existing codebase when adopting
\sys. 

\subsection{\sys Design}
\label{s:des:vl}

\PP{Synchronization using counters}
\sys has a \emph{64-bit atomic counter} at its core. The counter represents the
version of the \sys and it is monotonically increasing. The counter is
incremented atomically upon lock acquire and release and it only happens for
write lock. In \sys, acquire and release of read lock does involve updating the
counter; it incurs just a lookup to the counter. More details on the read and
write locks in \autoref{s:des:vl:APIs}. 

\PP{Concurrency model}
\sys supports non-blocking reads and exclusive writes. When the counter is even
it denotes that the lock is free and the writer acquires the lock by atomically
incrementing it. Therefore, if the counter is odd then it means that the lock is
currently held by the writer. So other writes spins on the counter until the
counter is back to even again. Lock is released by atomically incrementing the
counter once more (even-odd-even) and now any writers waiting on the lock would
try to acquire it. 

For reads, lock acquire involves looking up the current version number
(\emph{start\_version}) and taking a snapshot of it locally. Then the read
operation on the shared memory can be executed. Upon completion, lock release
operation verifies if the \emph{start\_version} has changed by comparing it with
global version counter \ie the current counter value.  A change in the version
denotes that a write operation has completed since the read started. This may
result in the read operation reading a stale or inconsistent value. In order,
to avoid this problem, the application has to retry the read operation by
calling lock acquire once again. This retry loop is delegated to the user side
code \ie the \sys read lock release just returns a bool value to signify if the
version number has changed. 

\subsubsection{\sys APIs}
\label{s:des:vl:APIs}



\autoref{f:api} shows the \sys APIs. \emph{write\_lock} is used to acquire lock
for write operation which internally manages the version counter. If the lock is
taken, then the call to \emph{write\_lock} keeps spinning until the lock is
available. On the other hand, the \emph{try\_write\_lock} API just checks if the
lock is available and acquires if so, upon lock unavailability it simply returns
false to the caller. If application needs asynchronous lock acquire operation it
can leverage the \emph{try\_write\_lock} API. The locks acquired using
\emph{write\_lock} and \emph{try\_write\_lock} can be released by calling
\emph{write\_unlock} which internally increments the counter to signify the lock
release. 

\begin{figure}[t]
% \vspace{-15px}
 \centering
 \begin{subfloat}
 \centering
 \inputminted[xleftmargin=11pt,fontsize=\scriptsize,escapeinside=@@]{c}{code/api.c}
 \end{subfloat}
 \coderule
 %\vspace{-5px}
	\caption{\sys APIs to be used directly in the application}
%\vspace{-20px}
 \label{f:api}
\end{figure}

For read operations, a call to \emph{read\_lock} is necessary to obtain the
current version number. The \emph{read\_lock} API is the non-blocking version
\ie, it does check the lock status and it simply returns the current version
number even if the lock is held by the writer during the call. On the other
hand, \emph{read\_no\_wait} API yields for the writer, \ie, if a writer a holds
the lock during the call, it waits until the lock is released and then acquires
the read lock by returning the new version number (the one after the write
completion). Applications can use either of two these APIs based on their
requirement. Read operations releases the lock by calling \emph{read\_unlock}
API which internally performs the version matching. Unlike the other APIs,
\emph{read\_unlock} API takes an argument which is nothing but the local copy of
the version number obtained by caller during the lock acquire operation. The
\emph{read\_unlock} functions simply compares this local copy with the current
version number from the counter. It returns true or false based on the
comparison operation; true signifies the read operation successful and a false
value means that the there was an concurrent write during the read operation and
hence this read operation is invalid. Upon receiving false value, application
can retry the critical section with calling \emph{read\_lock} again. 


\subsubsection{\sys Core API Implementation}
\label{s:des:vl:impl}

\autoref{f:vlw} and \autoref{f:vlr} shows the implementation of \sys's read and
write locking operations respectively. 

\PP{write\_lock}
Lines 1-26 in \autoref{f:vlw} shows the write\_lock implementation. Line 12-13
retrieves the version number and checks the lock status. Lines 16-22 is the lock
acquire logic. Line 19 performs a atomic CAS to switch the version number
(incrementing the version by 1). The \emph{retry\_lock\_acquire} will be
executed if the lock is already held or if the CAS fails. This retry loop makes
our write\_lock implementation a spin\_lock. 


\begin{figure}[t]
% \vspace{-15px}
 \centering
 \begin{subfloat}
 \centering
 \inputminted[xleftmargin=11pt,fontsize=\scriptsize,escapeinside=@@]{c}{code/vlw.c}
 \end{subfloat}
 \coderule
 %\vspace{-5px}
	\caption{Code snippet of \sys \emph{write\_lock} and \emph{write\_unlock} 
	functions. }
%\vspace{-20px}
 \label{f:vlw}
\end{figure}

\PP{write\_unlock}
Lines 28-39 in \autoref{f:vlw} shows the write\_unlock operation. Line 37
releases the lock by atomically incrementing the version number by 1. 
Lines 33-35 handles version number overflow \ie, it resets the version number to
0. 

\PP{read\_lock}
The read\_lock just returns the current version number (\autoref{f:vlr}). It is
the applications' responsibility to take store a local copy of this returned
version number. Because, this version number needs to be passed to relase the
read\_lock. Passing a wrong version number can result in a livelock situation. 


\begin{figure}[t]
% \vspace{-15px}
 \centering
 \begin{subfloat}
 \centering
 \inputminted[xleftmargin=11pt,fontsize=\scriptsize,escapeinside=@@]{c}{code/vlr.c}
 \end{subfloat}
 \coderule
 %\vspace{-5px}
	\caption{Code snippet of \sys \emph{read\_lock} and \emph{read\_unlock} 
	functions. }
%\vspace{-20px}
 \label{f:vlr}
\end{figure}

\PP{read\_unlock}
Lines 10-23 in \autoref{f:vlr} shows the read\_unlock code. Lines 17-18 compares
the local copy of the version number (passed by the caller) with the current
version number. If the comparison succeeds it returns true else false is
returned. 


\subsection{\sys Example Using a HashTable}
\label{s:des:eg}

\begin{figure}[t]
% \vspace{-15px}
 \centering
 \begin{subfloat}
 \centering
 \inputminted[xleftmargin=11pt,fontsize=\scriptsize,escapeinside=@@]{c}{code/vlht.c}
 \end{subfloat}
 \coderule
 %\vspace{-5px}
 \caption{ Code snippet of a \sys enabled
	hash table insert and lookup functions.} 
%\vspace{-20px}
 \label{f:vlht}
\end{figure}

\autoref{f:vlht}, illustrates \sys useage with a hashtable  example, 
Lines 1-13 illustrates the insert function. Line 6 gets the hash bucket and Line
7 acquires the per-bucket \sys.  Line 8 calls the list\_insert function which
inserts the node in the bucket chain. Line 12 unlocks the per-bucket
write\_lock.  

Lines 16-30, shows the lookup operation. Similar to the insert, Line 24 acquires
the per-bucket read\_lock and stores the local copy of version number in
\emph{version}. After the lookup operation in the Line 25, read\_unlock is
called on the bucket lock passing the local copy of the version number. If
read\_lock succeeds read operation ends on line 29, else the read operation
enters the retry loop. 






















