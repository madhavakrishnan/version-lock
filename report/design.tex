\section{Design}
\label{s:des}

- version lock design 
	- atomic counter for version
	- concurrency model
	- version lock APIs
- version lock pseudocode
- version lock hash table example
- correctness

In this section we explain our design goals followed by the \sys design.

\subsection{Design Goals}
\label{s:des:goals}

\PP{Avoid read-modify writes for read locks}

\PP{Avoid cacheline contention for read locks}

\PP{Achieve high scalability and performance}

\PP{Simple lock APIs and interface}

\begin{figure}[t]
% \vspace{-15px}
 \centering
 \begin{subfloat}
 \centering
 \inputminted[xleftmargin=11pt,fontsize=\scriptsize,escapeinside=@@]{c}{code/api.c}
 \end{subfloat}
 \coderule
 %\vspace{-5px}
	\caption{\sys APIs to be used directly in the application}
%\vspace{-20px}
 \label{f:vlw}
\end{figure}

\begin{figure}[t]
% \vspace{-15px}
 \centering
 \begin{subfloat}
 \centering
 \inputminted[xleftmargin=11pt,fontsize=\scriptsize,escapeinside=@@]{c}{code/vlw.c}
 \end{subfloat}
 \coderule
 %\vspace{-5px}
	\caption{Code snippet of \sys \emph{write\_lock} and \emph{write\_unlock} 
	functions. }
%\vspace{-20px}
 \label{f:vlw}
\end{figure}

\begin{figure}[t]
% \vspace{-15px}
 \centering
 \begin{subfloat}
 \centering
 \inputminted[xleftmargin=11pt,fontsize=\scriptsize,escapeinside=@@]{c}{code/vlr.c}
 \end{subfloat}
 \coderule
 %\vspace{-5px}
	\caption{Code snippet of \sys \emph{read\_lock} and \emph{read\_unlock} 
	functions. }
%\vspace{-20px}
 \label{f:vlr}
\end{figure}

\begin{figure}[t]
% \vspace{-15px}
 \centering
 \begin{subfloat}
 \centering
 \inputminted[xleftmargin=11pt,fontsize=\scriptsize,escapeinside=@@]{c}{code/vlht.c}
 \end{subfloat}
 \coderule
 %\vspace{-5px}
 \caption{ Code snippet of a \sys enabled
	hash table insert and lookup functions.} 
%\vspace{-20px}
 \label{f:vlht}
\end{figure}

