\section{Evaluation}
\label{s:eval}

In this work, we compare pessimistic concurrency control (PCC) and optimistic
concurrency control (OCC) using a hash table. We implement variants of an open
chaining hash table that only differ in the underlying synchronization primitive
used. For PCC, we use a reader-writer lock for each bucket of the hash table to allow
concurrent threads working on disjoint buckets to proceed in parallel. In case of
OCC, we have two variants : a lock-free hash table and another based on
version locking. These variants use atomic instructions such as 
\textit{fetch-and-add} and \textit{compare-and-swap} for updating items in the
hash table while reading an item may involve retries to ensure consistency of the
read operation.

Our evaluation is aimed at answering the following questions:

\squishlists

\item Can version locking outperform lock-free and a reader-writer lock-based
design of the hash table ? Does the performance scale with increasing thread
counts ? (\autoref{s:eval:scale})

\item What is the impact of data hotspots on the performance of different
hash table variants ? (\autoref{s:eval:skew})

\item What is the performance implication of using read-modify-write primitives
for read-only operations, as done in the case of reader-writer locks ? 
(\autoref{s:eval:rmw-read})

\item How the performance varies when the computation and memory are physically
separated across different NUMA domains ? (\autoref{s:eval:numa})

\squishends

\subsection{Evaluation Setup}
\label{s:eval:env}

\begin{table}[t!]
    \ra{1.1}
    \centering
    %\scriptsize
    %\small
    \footnotesize
    \resizebox{\columnwidth}{!}{%\begin{tabular}{|c|l|}
\newcolumntype{C}[1]{>{\centering\let\newline\\\arraybackslash\hspace{0pt}}m{#1}}
\newcolumntype{L}[1]{>{\raggedright\let\newline\\\arraybackslash\hspace{0pt}}m{#1}}
\newcolumntype{R}[1]{>{\raggedleft\let\newline\\\arraybackslash\hspace{0pt}}m{#1}}

\begin{tabular}{|C{0.20\columnwidth}|L{0.60\columnwidth}|}
    \hline
    \textbf{Workloads} & \multicolumn{1}{c|}{\textbf{Characteristics}} \\ \hline
    YCSB-A            & Write-intensive: 50\% Writes and 50\% Reads \\ \hline
    YCSB-B            & Read-intensive: 5\% Writes and 95\% Reads   \\ \hline
    YCSB-C            & Read-only: 100\% Reads                       \\ \hline
    YCSB-D            & Read-latest: 5\% Writes and 95\% Reads      \\ \hline
\end{tabular}
}
    \vspace{5px}
    \caption{
	YCSB workload characteristics.
    }
    \label{s:ycsb}
    \vspace{-5pt}
\end{table}

We ran our evaluations on a system with Intel(R) Xeon(R) Gold 5218 2.3 GHz CPU that has
two NUMA nodes and 16 cores (32 hyperthreads) per node. The system is running Fedora 29
on Linux kernel 4.18.16. We use YCSB~\cite{ycsb-cooper-socc10} -- a standard key-value
store benchmark for evaluating the hash table variants. YCSB provides various read-write
ratios for different workloads, see~\autoref{s:ycsb}. We use
index-microbench~\cite{indexbench-Wang-sigmod18} to generate the YCSB workloads.
We use both uniform and Zipfian distribution with random integer keys in our
workloads. For all evaluations, we first populate the hash table with 64 million
items and then run the workloads, which performs 64 million operations.

\begin{figure}[t]
    \centering
    \input{data/tput-uniform}
    \vspace{-20pt}
    \caption{Performance comparison for uniform workloads.}
    \label{f:unif}
    \vspace{-2pt}
\end{figure}

\subsection{Scalability}
\label{s:eval:scale}
In this section, we evaluate the scalability of the different concurrency control
mechanisms using uniform workloads wherein each item has an equal probability of
being accessed. \autoref{f:unif} shows the throughput for the hash table variants in 
million operations per second (Mops/sec) for the various workloads as we vary
the number of threads. For the write-intensive YCSB-A workload (\autoref{f:unif}(a)),
version locking delivers up to 10\% higher throughput than both the lock-free and
reader-writer lock (rw-lock) for all thread counts except at 64 threads. Better
performance than rw-lock is due to no read-modify-writes required for the 50\% read
operations in this workload. Also, version locking has better performance than lock-free
since we use an 8-byte compare-and-swap (CAS) while lock-free uses a 16-byte CAS to
complete read operations. However, at 64 threads, rw-lock outperforms both lock-free and
version locking approaches. Our evaluation machine has 32 cores (64 hyperthreads) across
the two NUMA nodes, so we believe a simple serialization for rw-lock in comparison to
retrying as done in version locking and lock-free is the primary reason for this behaviour.
Employing exponential backoff to limit retrying can possibly help to limit performance
degradation, although we have not investigated it in more detail.

Other workloads are read-intensive and read-mostly, and we see better
performance for version locking and lock-free as compared to rw-lock. This is expected since
readers still need to acquire and release the shared lock for reading an item from the
hash table incurring memory writes in the process. For YCSB-B, the performance gap between
version locking and rw-lock is 46\% at 16 threads. Upon increasing the number of threads
further, the improvement decreases as all threads are utilized in our system. For the
read-only YCSB-C workload (\autoref{f:unif}(c)), version locking achieves higher
throughput than other variants for all thread counts including at 64 threads where it
outperforms lock-free and rw-lock by 5\% and 21\%, respectively. The performance trend for
YCSB-D is similar to YCSB-B, so we skip discussing it in more detail and exclude it in our
later evaluations.


\begin{figure*}[t]
    \centering
    \input{data/tput-zipf}
    %\vspace{-20pt}
    \caption{Performance comparison for Zipfian workloads.}
    \label{f:zipf}
    \vspace{-8pt}
\end{figure*}

\subsection{Performance under skewed workloads}
\label{s:eval:skew}
We now evaluate our hash table variants with skewed workloads, generated from
a Zipfian distribution (Zipfian coefficient = 0.99). Skewed workloads are common
in production environments, as noted in ~\cite{ycsb-cooper-socc10, hotring-chen-fast20}.
The results are shown in \autoref{f:zipf}.

For YCSB-A, the performance improvement for version locking over others is smaller
(between 4-10\%) compared to uniform workloads. For skewed workloads, some key-value
pairs are accessed with high probability, so contention is not spread as evenly throughout
the hash table. As a result, the probability of retries for both read and write
operations for the OCC variants is higher under workload skew. Still, version locking
has better performance than lock-free and rw-lock, except at 64 threads where rw-lock
achieves the highest throughput. 

Under read-mostly and read-only (\autoref{f:zipf}(b),(c)) workloads, OCC variants
always outperform rw-lock. Also, version locking has similar or better performance
than lock-free design for each configuration.

\subsection{Optimism vs Pessimism for read operations}
\label{s:eval:rmw-read}
In this section, we want to compare the optimistic and pessimistic approach for
read operations and discuss their impact on performance of the application. Specfically,
we want to look at performance of read-only operations for both uniform(\autoref{f:unif}(c))
and Zipfian workloads(\autoref{f:zipf}(c)). In each case, OCC variants comprehensively
outperform pessimitic locks as they perform read operations without writing to memory.
Hence, threads reading the same or different items do not conflict with each other.
In contrast, when using a rw-lock, a shared lock is acquired before initiating the read
and upon its completion, the lock is released. As a result, performing a read operation
involves read-modify-write primitives that cause cacheline invalidations, resulting
in performance degradation. Hence, OCC is better suited for read-intensive workloads.
While this project doesn't deal with workload characterization, we believe insights
from such studies will help in making informed choices regarding the best suited
concurrency control in an application.

\begin{figure*}[t]
    \centering
    \input{data/tput-numa}
    %\vspace{-20pt}
    \caption{Performance under NUMA separation.}
    \label{f:numa}
    \vspace{-8pt}
\end{figure*}

\subsection{Compute-Memory Separation}
\label{s:eval:numa}
In our prior evaluations, we use all hyperthreads in our machine and do not consider
the placement of hash table in memory and where the threads accessing the hash table
are running. By default, Linux tries to allocate memory on the same NUMA node as the
node where the thread is running, so locality of compute and memory is achieved
out-of-the-box. Now, we want to compare the performance of the hash table variants
when the compute and memory are separated across different NUMA nodes. We pin threads
on one NUMA node while the entire hash table resides on another NUMA node and run
uniform workloads, similar to \autoref{s:eval:scale}. Our motivation is to study
the impact of locality on the performance of the hash table variants and how the
trends deviate from our prior results. The results are shown in \autoref{f:numa}.

For the write-intensive YCSB-A(\autoref{f:numa}(a)), version locking and lock-free
have similar performance across all thread counts and the throughput is only 
slightly better than rw-lock up to 8 threads. As we increase the number of threads,
all variants have similar performance. This is a deviation from our prior evaluation
of YCSB-A for both uniform and Zipfian workloads where version locking was able to
significantly outperform rw-lock. Accessing remote memory is expensive compared to
local memory, thus an optimized concurrency control does not translate into 
performance boost for the application. The key insight from this evaluation is that
the application performance is heavily impacted by the locality of compute and memory,
independent of specialization done within the application.

For workloads involving mostly read operations(\autoref{f:numa}(b),(c)), performance
for all hash table variants is lower compared to equivalent settings when remote
memory pinning is not done(\autoref{f:unif}(b),(c)) since accessing remote memory
has higher latency than local memory. However, version locking still achieves
better performance than other variants due to limited or lack of contention for its
operations.


