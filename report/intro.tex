\section{Introduction}
\label{s:intro}

The introduction of multi-core machines marked the end of the \emph{free lunch}
~\cite{herb-sutter, shuffl-locks} making concurrent programming indispensable 
when designing a high-performance system software or applications. The lock
based concurrency control has been evolving for the last two decades and it has
been widely used in the many critical systems software such as the Linux
Kernel~\cite{}, database systems~\cite{}, web servers and applications~\cite{},
cloud and datacenter applications\cite{} etc.

The concurrency control can be categorized as 1) pessimistic concurrency control
(PCC) which proactively acquires lock before modifying the shared memory, and
2) optimistic concurrency control (OCC) which modifies the shared memory and
resolves inconsistency by performing conflict resolution. The most commonly
available locks such as the Mutex~\cite{}, Readers-writer (RW) lock are examples
of PCC. Two-phase locking (2PL)~\cite{} and lock-free programming~\cite{} are
common examples of the OCC. 2PL is most commonly used in the database
transactions and not see as a viable candidate for designing low level data
structures. On the other hand, lock-free programming~\cite{} are used to 
design common low-level data structures such as a linked list, hash table, 
binary search tree etc. Lock-free programming have also been used in the 
prior research to design more complex data structures such as a B+tree~\cite{}. 

Overall, PCC are easy to use and reason about the correctness but this comes at
a poor scalability and performance with the increasing core count. On the
contrary, OCC particularly the lock-free programming 
enables better
scalability but it is notoriously hard to program even the simple data
structures. Writing complex data structures such as B+tree requires some serious
engineering effort if not impossible. Further, lock-free programming is also
prone to ABA problem~\cite{} which it even more harder. We explain more 
on OCC and PCC in \autoref{s:bg}.

So the key question is, \emph{can we make locks optimistic (for scalability) while
retaining the simplicity of the PCC?}. We propose \sys as answer to this
question. \sys is an OCC technique that supports non-blocking reads and
exclusive writes. \sys is designed with scalability, performance, and simplicity
in mind. 
We designed \sys using C++ and we expose the \sys with a simple set of
APIs as found in the C++ standard library and POSIX library. We designed a
hashtable with \sys and evaluated it using the YCSB workloads~\cite{}. We
compare \sys's performance with the RW lock (PCC) and lock-free hashtable (OCC).
Over results show that, \sys outperforms RW lock by up to 40\% and it performs
on par or better than lock-free while being much simple to use than lock-free.  
